\documentclass[10pt,a4paper]{article} 
% Lưu ý: Size chữ cơ bản (11pt) và khổ giấy (a4paper) nên chỉnh trực tiếp ở dòng trên.

% =================================================================
% --- 1. USER CONFIGURATION (CẤU HÌNH SIÊU THAM SỐ) ---
% =================================================================
% --- A. CẤU HÌNH LỀ TRANG (PAGE MARGINS) ---
% Đơn vị hỗ trợ: in (inch), cm, mm, pt.
\def\PageLeftMargin{0.59in}   % Lề trái
\def\PageRightMargin{0.59in}  % Lề phải (Nên bằng lề trái để căn giữa đẹp)
\def\PageTopMargin{0.3in}     % Lề trên
\def\PageBottomMargin{0.3in}  % Lề dưới

% --- B. CẤU HÌNH KÍCH THƯỚC & KHOẢNG CÁCH ---
\newlength{\NumberWidth}
\setlength{\NumberWidth}{1.25cm} % Độ rộng dành cho số thứ tự Heading (VD: 1.1.)

\newlength{\LogoWidth}
\setlength{\LogoWidth}{4cm}      % Độ rộng của Logo ở trang bìa

\def\GlobalParSkip{6pt}          % Khoảng cách giữa các đoạn văn

% --- C. CẤU HÌNH MÀU SẮC (RGB) ---
\def\BrandColorRGB{237, 125, 49} % Màu chủ đạo (Cam)
\def\TableHeadColor{brandOrange} % Màu nền tiêu đề bảng (tham chiếu tên màu đã tạo)

% --- D. CẤU HÌNH FONT CHỮ ---
\def\PreferredFont{Arial}           
\def\FallbackFont{TeX Gyre Heros}   

% --- E. THÔNG TIN CHÂN TRANG (FOOTER INFO) ---
\def\CompanyCopyright{Bestarion} % Tên công ty ở Footer
\def\CopyrightYear{2025}         % Năm bản quyền


% =================================================================
% --- 2. PACKAGES & BASIC SETUP (KHÔNG CẦN CHỈNH SỬA TỪ ĐÂY) ---
% =================================================================
% Quan trọng: Nạp fancyhdr sớm để tránh lỗi undefined control sequence
\usepackage{fancyhdr} 
\usepackage[utf8]{inputenc}
\usepackage[english]{babel}
\usepackage{fontspec}
\usepackage{setspace}
\usepackage{calc}
\usepackage{etoolbox}
\usepackage[table]{xcolor} 


\usepackage{graphicx}
\makeatletter
\providecommand{\pandocbounded}[1]{#1}
\makeatother

% Áp dụng cấu hình margin
\usepackage[
  left=\PageLeftMargin,
  right=\PageRightMargin,
  top=\PageTopMargin, 
  bottom=\PageBottomMargin
]{geometry}

% Cấu hình Font
\IfFontExistsTF{\PreferredFont}{
    \setmainfont{\PreferredFont}
}{
    \setmainfont{\FallbackFont}
}

% Áp dụng khoảng cách đoạn
\setlength{\parindent}{0pt}   
\setlength{\parskip}{\GlobalParSkip}     


\providecommand{\tightlist}{%
  \setlength{\itemsep}{0pt}\setlength{\parskip}{0pt}}

% =================================================================
% --- 3. COLORS & GRAPHICS ---
% =================================================================
\usepackage{graphicx}
\usepackage{float}

% Định nghĩa màu từ tham số RGB
\definecolor{brandOrange}{RGB}{\BrandColorRGB}
\definecolor{mediumOrange}{RGB}{250, 206, 156} 
\definecolor{lightOrange}{RGB}{253, 226, 202} 

\makeatletter
\def\maxwidth{\ifdim\Gin@nat@width>\linewidth\linewidth\else\Gin@nat@width\fi}
\def\maxheight{\ifdim\Gin@nat@height>\textheight\textheight\else\Gin@nat@height\fi}
\makeatother
\setkeys{Gin}{width=\maxwidth, height=\maxheight, keepaspectratio}

% Fix hình ảnh căn giữa
\let\origfigure\figure
\let\endorigfigure\endfigure
\renewenvironment{figure}[1][H] {\origfigure[#1]\centering} {\endorigfigure}

% =================================================================
% --- 4. HEADER & FOOTER STYLING ---
% =================================================================
\pagestyle{fancy}
\fancyhf{} 

% Mở rộng Header sang trái
\fancyhfoffset[L]{\NumberWidth}

% Header
    \fancyhead[L]{\fontfamily{ptm}\selectfont\textcolor{gray}{\small Bestarion.QM.CON.001-Controlled\_Document\_Convention-2\_1}\vspace{3pt}}

% Footer (Sử dụng biến tham số hóa)
\fancyfoot[L]{\fontfamily{ptm}\selectfont\textcolor{gray}{\small Copyright \textcopyright\ \CopyrightYear, \CompanyCopyright.}}
\fancyfoot[R]{\fontfamily{ptm}\selectfont\thepage}

% Rules
\renewcommand{\headrulewidth}{0.4pt}
\setlength{\headheight}{15pt} 
\renewcommand{\headrule}{\hbox to\headwidth{\color{gray}\leaders\hrule height \headrulewidth\hfill}}
\renewcommand{\footrulewidth}{0pt} 

% =================================================================
% --- 5. HEADING STYLES ---
% =================================================================
\usepackage{titlesec}

\newcommand{\hangnum}[1]{\llap{\makebox[\NumberWidth][l]{#1}}}

\titlespacing*{\section}{0pt}{12pt}{6pt}
\titlespacing*{\subsection}{0pt}{10pt}{4pt}
\titlespacing*{\subsubsection}{0pt}{8pt}{3pt}

\titleformat{\section}[hang]
  {\color{brandOrange}\normalfont\Large\bfseries} 
  {\hangnum{\thesection.}} 
  {0pt} {}

\titleformat{\subsection}[hang]
  {\color{black}\normalfont\large\bfseries}
  {\hangnum{\thesubsection.}} 
  {0pt} {}

\titleformat{\subsubsection}[hang]
  {\color{black}\normalfont\normalsize\bfseries}
  {\hangnum{\thesubsubsection.}} 
  {0pt} {}

% =================================================================
% --- 6. LIST STYLES ---
% =================================================================
\usepackage{enumitem}
\usepackage{amssymb} 

\setlist{nosep, leftmargin=*} 
\setlist[itemize,1]{label=\raisebox{-0.9ex}{\textcolor{black}{\Huge\textbullet}}}
\setlist[itemize,2]{label=\textcolor{black}{\tiny$\blacksquare$}}
\setlist[itemize,3]{label=\textbullet}

% =================================================================
% --- 7. TABLES & CAPTIONS ---
% =================================================================
\usepackage{array}
\usepackage{longtable}
\usepackage{caption}

\captionsetup{
  labelfont={bf, color=black},
  textfont={it},
  labelsep=colon,
  skip=10pt,
  position=bottom 
}

\newcommand{\globalTableCaption}{}
\AtBeginEnvironment{longtable}{
  \renewcommand{\globalTableCaption}{} 
  \let\OldCaption\caption 
  \renewcommand{\caption}[2][]{
     \renewcommand{\globalTableCaption}{##2}
     \addcontentsline{lot}{table}{\numberline{\thetable}##2}
  }
}
\AtEndEnvironment{longtable}{
  \ifdefempty{\globalTableCaption}{}{
    \begin{center} \captionof{table}{\globalTableCaption} \end{center}
  }
}

% =================================================================
% --- 8. TOC & HYPERLINKS ---
% =================================================================
\usepackage{tocloft}
\usepackage[unicode=true]{hyperref}

\tocloftpagestyle{fancy} 

\renewcommand{\cfttoctitlefont}{%
    \hspace{-\NumberWidth}%
    \color{brandOrange}\Huge\bfseries\sffamily\MakeUppercase
}
\renewcommand{\cftlottitlefont}{%
    \hspace{-\NumberWidth}%
    \color{brandOrange}\Huge\bfseries\sffamily\MakeUppercase
}
\renewcommand{\cftloftitlefont}{%
    \hspace{-\NumberWidth}%
    \color{brandOrange}\Huge\bfseries\sffamily\MakeUppercase
}

\setlength{\cftsecindent}{-\NumberWidth}    
\setlength{\cftsubsecindent}{0pt}           
\setlength{\cftsubsubsecindent}{1.5em}      

\hypersetup{
    colorlinks=true, linkcolor=blue, filecolor=magenta, urlcolor=blue,
    pdftitle={Controlled Document Convention}, pdfpagemode=FullScreen,
}

\newcommand{\insertTOC}{\newpage \renewcommand{\contentsname}{TABLE OF CONTENT} \tableofcontents \newpage}
\newcommand{\insertListTables}{\newpage \renewcommand{\listtablename}{INDEX OF TABLES} \listoftables \newpage}
\newcommand{\insertListFigures}{\newpage \renewcommand{\listfigurename}{INDEX OF FIGURES} \listoffigures \newpage}

% =================================================================
% --- 9. TITLE PAGE (TRANG BÌA) --- [FIXED CENTER]
% =================================================================
\renewcommand{\maketitle}{
    \thispagestyle{empty}
    
    \begin{titlepage}
        % Cấu hình lề cho trang bìa (sử dụng cm cho chuẩn xác hoặc đổi sang inch tùy ý)
        \newgeometry{
            top=3cm, bottom=3cm, 
            left=2.5cm, right=2.5cm
        }
        
        \setlength{\parindent}{0pt}
        \centering 
        
        \vspace*{1cm}
        
        % Logo (Sử dụng biến LogoWidth)
        \IfFileExists{logo.jpeg}{\includegraphics[width=\LogoWidth]{logo.jpeg}}{
            \IfFileExists{logo.png}{\includegraphics[width=\LogoWidth]{logo.png}}{
                \begin{center}\fbox{\parbox[c][2cm]{4cm}{\centering LOGO MISSING}}\end{center}
            }
        }
        \vspace{0.5cm} 
        
        % Title
        {\color{brandOrange} \fontsize{30}{75}\selectfont CONTROLLED
DOCUMENT CONVENTION \par}
        
        % Subtitle
         \vspace{1.3cm} {\fontsize{24}{28}\selectfont QM \par} 
        
        \vfill 
        
        % Metadata Table
        \begin{minipage}{0.8\textwidth} 
            \centering 
            \raggedright
            \renewcommand{\arraystretch}{1.5}
            \begin{tabular}{ll}
                \textbf{Security classification:} & \detokenize{INTERNAL} \\
                \textbf{Document code:} & \detokenize{Bestarion.QM.CON.001} \\
                \textbf{Last updated by:} & \detokenize{Khoa
Nguyen-Dinh} \\
                \textbf{Effective date:} & \detokenize{Aug 09, 2023} \\
                \textbf{Version:} & \detokenize{2.1} \\
                \textbf{Template ID:} & \detokenize{ODT\_Base\_Template} \\
            \end{tabular}
        \end{minipage}
        
        \restoregeometry
    \end{titlepage}
}

% =================================================================
% --- 10. DOCUMENT CONTENT --- 
% =================================================================
\onehalfspacing
\begin{document}

\maketitle 
\clearpage 

% --- Document Control Table ---
\noindent\hspace{-\NumberWidth}%
{\color{brandOrange}\fontsize{15}{18}\selectfont\bfseries DOCUMENT CONTROL}%
\par\vspace{6pt}

\begin{longtable}{|p{0.09\textwidth}|p{0.22\textwidth}|p{0.14\textwidth}|p{0.12\textwidth}|p{0.16\textwidth}|p{0.12\textwidth}|}
    \hline
    \rowcolor{\TableHeadColor} % Sử dụng biến màu
    \textbf{Version} & \textbf{Change Description} & \textbf{Changed By} & \textbf{Date} & \textbf{Approved By} & \textbf{Date} \\
    \hline
        1.0 & Khởi tạo tài liệu và phê duyệt lần
đầu. & KhoaND & 2025-01-01 & CEO/TienDD & 2025-01-05 \\ \hline
        1.1 & Cập nhật phần quy trình
3.2. & MinhNT & 2025-02-10 & N/A & N/A \\ \hline
    \end{longtable}
\clearpage 

% --- TOC ---
\insertTOC  
\insertListTables  
\insertListFigures

% Reset bộ đếm
\setcounter{table}{0}
\setcounter{figure}{0}

% --- Main Body ---
\section{\texorpdfstring{\textbf{Introduction}}{Introduction}}\label{introduction}

\subsection{\texorpdfstring{\textbf{Purpose}}{Purpose}}\label{purpose}

This document specifies conventions for naming and version of documents
which are in scope of QMS \& ISMS.

\subsection{\texorpdfstring{\textbf{Scope}}{Scope}}\label{scope}

This convention is applied to only the controlled documents, which are
internally created by Bestarion.

Following documents are outside the scope of this convention:

\begin{itemize}
\tightlist
\item
  Uncontrolled documents and records
\item
  Documents obtained from external sources (such as, government
  regulatory,\ldots)
\item
  Project documents to follow customers' conventions, regarding naming
  and/ or version
\end{itemize}

\subsection{\texorpdfstring{\textbf{Definition And
Abbreviation}}{Definition And Abbreviation}}\label{definition-and-abbreviation}

\begin{itemize}
\item
  BoD: Board of Directors
\item
  QM: Quality Management
\item
  Controlled Document: A document with its versions to be controlled, in
  order to ensure appropriate
\end{itemize}

access to correct content. When a controlled document is to be changed,
its new

version must be controlled via appropriate review and approval
procedures

Document: Information and its supporting medium (text files, images,
video, voice,\ldots)

Record: Is a kind of document which is resulted from performing
operations

\subsection{\texorpdfstring{\textbf{References}}{References}}\label{references}

\begin{itemize}
\tightlist
\item
  {[}1{]} Document\_Responsibility\_Authority\_Standard
\item
  {[}2{]} Control\_Of\_Documented\_Information
\end{itemize}

\section{\texorpdfstring{\textbf{Regulation}}{Regulation}}\label{regulation}

The objective for document owner when creating a new document is no more
than 1 easily identifiable error in 5 A4 pages

\section{\texorpdfstring{\textbf{Naming Of Controlled Non-record
Documents}}{Naming Of Controlled Non-record Documents}}\label{naming-of-controlled-non-record-documents}

When a controlled document is not a record, the file name shall follow
below format:

\textbf{{[}Document Code{]}-{[}Document Name{]}-{[}Version Number{]}}

Each part is separated by a hyphen.

\subsection{\texorpdfstring{\textbf{Document
Code}}{Document Code}}\label{document-code}

The document code can be used as a short name to refer to the document.

Document Code is constructed by following format:

\textbf{{[}Unit Code{]}.{[}Document Type Code{]}.{[}Ordinal Number{]}}

Each part is separated by a stop (`.').

\begin{itemize}
\tightlist
\item
  \textbf{Unit Code}: Code representing the unit

  \begin{itemize}
  \tightlist
  \item
    In case unit is a department or portfolio: refer to corporate
    organizational structure for the corresponding code

    \begin{itemize}
    \tightlist
    \item
      Bestarion

      \begin{itemize}
      \tightlist
      \item
        IT: Information Technology Department
      \end{itemize}
    \end{itemize}
  \item
    In case unit is a sub-group within a department or a portfolio:
    refer to the department/ portfolio charter for the corresponding
    code
  \item
    In case document apply in a sub-company, unit code may be included
    company name
  \end{itemize}
\item
  \textbf{Document Type Code}: Refer to \hyperref[page-5-8]{{[}1{]}} for
  the codes of document types
\item
  \textbf{Ordinal Number} (3 digits): A unique number to differentiate
  documents in the same document type, belonging to the same unit
\end{itemize}

\emph{Examples: HR.GUI.234, QM.POL.001},
\emph{Bestarion.IT.POL.001}\ldots{}

\subsection{\texorpdfstring{\textbf{Document
Name}}{Document Name}}\label{document-name}

Document name is a unique name of document, with following convention:

\begin{itemize}
\tightlist
\item
  The first letter of each word shall be in capital
\item
  Words are separated by underscore character (`\_')
\end{itemize}

\emph{Examples: Project\_Planning, Risk\_Management.}

Document name should not include the type of document, since the type is
already known in Document Code part \hyperref[page-5-1]{(3.1.)}

\subsection{\texorpdfstring{\textbf{Version
Number}}{Version Number}}\label{version-number}

A version number is constructed of 3 numbers: ``Major version number'',
``Minor version number'' and ``Draft number''.

Those numbers are separated by a underscore (``\_'').

\textbf{{[}Major Version Number{]}\_{[}Minor Version Number{]}\_{[}Draft
Number{]}}

\begin{itemize}
\tightlist
\item
  \textbf{Major version number}: Started from 0. The first baseline has
  major version equals 1. Major version is increased when there are
  changes to the document content, which affect to the original meaning
  of the document
\item
  \textbf{Minor version number}: Started from 0. Minor version is
  increased when there are only changes to enhance or correct the
  content, while the original meaning is intact
\item
  \textbf{Draft number}: Started from 1. This number is increased when a
  draft is updated and re-submitted for review. A document may pass
  through several draft updates until it is baseline
\item
  \textbf{Note:} Before the first baseline of each document, draft
  version number for document is constructed of 2 numbers: ``Major
  version number'' and ``Draft number''
\item
  \emph{Examples:}

  \begin{itemize}
  \tightlist
  \item
    \emph{QM.POL.001-Quality\_Manual-0\_1 The first draft of policy
    `Quality Manual', issued by QM}
  \item
    \emph{QM.POL.001-Quality\_Manual-0\_2 The second draft, re-submitted
    for review}
  \end{itemize}
\end{itemize}

\begin{table}[h]
\centering
{\renewcommand{\arraystretch}{1.5}
\begin{tabular}{|l|l|l|}
\hline

\rowcolor{brandOrange}\textbf{} & \textbf{QM.POL.001-Quality\_Manual-1\_0} & \textbf{The first baseline of policy ‘Quality Manual’, issued by QM} \\
\hline

 & QM.POL.001-Quality\_Manual-1\_0\_1 & The first draft version after the first baseline \\

\hline

 & QM.POL.001-Quality\_Manual-1\_1 & The minor baseline of the document, based on version 1.0 \\

\hline

 & QM.POL.001-Quality\_Manual-2\_0 & The major baseline of the document, based on version 1.x \\

\hline

\end{tabular}}

\caption{Test figure}

\end{table}

\begin{figure}
\centering
\pandocbounded{\includegraphics[keepaspectratio,alt={Hinh so 1}]{./_page_13_Picture_7.jpeg}}
\caption{Hinh so 1}
\end{figure}

\begin{itemize}
\tightlist
\item
  aouwhfuahfuawhfuawd
\item
  Sua lan 4
\end{itemize}

\begin{figure}
\centering
\pandocbounded{\includegraphics[keepaspectratio,alt={Hinh so 2}]{./_page_14_Figure_1.jpeg}}
\caption{Hinh so 2}
\end{figure}

\begin{figure}
\centering
\pandocbounded{\includegraphics[keepaspectratio,alt={Hinh so 3}]{./image.png}}
\caption{Hinh so 3}
\end{figure}

\end{document}
