\documentclass[11pt,a4paper]{article}

% =================================================================
% --- 1. USER CONFIGURATION (CẤU HÌNH SIÊU THAM SỐ) ---
% =================================================================
\def\PageLeftMargin{2.5cm}   % Lề trái (Rộng hơn để chứa số thứ tự)
\def\PageRightMargin{2.5cm}  % Lề phải (Cân bằng để fix lỗi căn giữa)
\def\PageTopMargin{1.5cm}    
\def\PageBottomMargin{1.5cm} 


\newlength{\NumberWidth}
\setlength{\NumberWidth}{1.25cm} 

\newlength{\LogoWidth}
\setlength{\LogoWidth}{4cm}      % Độ rộng của Logo ở trang bìa
\def\GlobalParSkip{6pt}          % Khoảng cách giữa các đoạn văn
% --- C. CẤU HÌNH MÀU SẮC (RGB) ---
\def\BrandColorRGB{237, 125, 49} % Màu chủ đạo (Cam)
\def\TableHeadColor{brandOrange} % Màu nền tiêu đề bảng (tham chiếu tên màu đã tạo)

\def\CompanyCopyright{Bestarion} % Tên công ty ở Footer
\def\CopyrightYear{2025} 

\def\BrandColorRGB{237, 125, 49} 
\def\PreferredFont{Arial}           
\def\FallbackFont{TeX Gyre Heros}   

% =================================================================
% --- 2. PACKAGES & BASIC SETUP ---
% =================================================================
% Quan trọng: Nạp fancyhdr sớm để tránh lỗi undefined control sequence
\usepackage{fancyhdr} 
\usepackage[utf8]{inputenc}
\usepackage[english]{babel}
\usepackage{fontspec}
\usepackage{setspace}
\usepackage{calc}
\usepackage{etoolbox}
\usepackage[table]{xcolor} % Nạp xcolor với option table

\usepackage{graphicx}
\makeatletter
\providecommand{\pandocbounded}[1]{#1}
\makeatother


\usepackage[
  left=\PageLeftMargin,
  right=\PageRightMargin,
  top=\PageTopMargin, 
  bottom=\PageBottomMargin
]{geometry}

\IfFontExistsTF{\PreferredFont}{
    \setmainfont{\PreferredFont}
}{
    \setmainfont{\FallbackFont}
}

\setlength{\parindent}{0pt}   
\setlength{\parskip}{\GlobalParSkip}    

\usepackage{color}
\usepackage{fancyvrb}
\newcommand{\VerbBar}{|}
\newcommand{\VERB}{\Verb[commandchars=\\\{\}]}
\DefineVerbatimEnvironment{Highlighting}{Verbatim}{commandchars=\\\{\}}
% Add ',fontsize=\small' for more characters per line
\newenvironment{Shaded}{}{}
\newcommand{\AlertTok}[1]{\textcolor[rgb]{1.00,0.00,0.00}{\textbf{#1}}}
\newcommand{\AnnotationTok}[1]{\textcolor[rgb]{0.38,0.63,0.69}{\textbf{\textit{#1}}}}
\newcommand{\AttributeTok}[1]{\textcolor[rgb]{0.49,0.56,0.16}{#1}}
\newcommand{\BaseNTok}[1]{\textcolor[rgb]{0.25,0.63,0.44}{#1}}
\newcommand{\BuiltInTok}[1]{\textcolor[rgb]{0.00,0.50,0.00}{#1}}
\newcommand{\CharTok}[1]{\textcolor[rgb]{0.25,0.44,0.63}{#1}}
\newcommand{\CommentTok}[1]{\textcolor[rgb]{0.38,0.63,0.69}{\textit{#1}}}
\newcommand{\CommentVarTok}[1]{\textcolor[rgb]{0.38,0.63,0.69}{\textbf{\textit{#1}}}}
\newcommand{\ConstantTok}[1]{\textcolor[rgb]{0.53,0.00,0.00}{#1}}
\newcommand{\ControlFlowTok}[1]{\textcolor[rgb]{0.00,0.44,0.13}{\textbf{#1}}}
\newcommand{\DataTypeTok}[1]{\textcolor[rgb]{0.56,0.13,0.00}{#1}}
\newcommand{\DecValTok}[1]{\textcolor[rgb]{0.25,0.63,0.44}{#1}}
\newcommand{\DocumentationTok}[1]{\textcolor[rgb]{0.73,0.13,0.13}{\textit{#1}}}
\newcommand{\ErrorTok}[1]{\textcolor[rgb]{1.00,0.00,0.00}{\textbf{#1}}}
\newcommand{\ExtensionTok}[1]{#1}
\newcommand{\FloatTok}[1]{\textcolor[rgb]{0.25,0.63,0.44}{#1}}
\newcommand{\FunctionTok}[1]{\textcolor[rgb]{0.02,0.16,0.49}{#1}}
\newcommand{\ImportTok}[1]{\textcolor[rgb]{0.00,0.50,0.00}{\textbf{#1}}}
\newcommand{\InformationTok}[1]{\textcolor[rgb]{0.38,0.63,0.69}{\textbf{\textit{#1}}}}
\newcommand{\KeywordTok}[1]{\textcolor[rgb]{0.00,0.44,0.13}{\textbf{#1}}}
\newcommand{\NormalTok}[1]{#1}
\newcommand{\OperatorTok}[1]{\textcolor[rgb]{0.40,0.40,0.40}{#1}}
\newcommand{\OtherTok}[1]{\textcolor[rgb]{0.00,0.44,0.13}{#1}}
\newcommand{\PreprocessorTok}[1]{\textcolor[rgb]{0.74,0.48,0.00}{#1}}
\newcommand{\RegionMarkerTok}[1]{#1}
\newcommand{\SpecialCharTok}[1]{\textcolor[rgb]{0.25,0.44,0.63}{#1}}
\newcommand{\SpecialStringTok}[1]{\textcolor[rgb]{0.73,0.40,0.53}{#1}}
\newcommand{\StringTok}[1]{\textcolor[rgb]{0.25,0.44,0.63}{#1}}
\newcommand{\VariableTok}[1]{\textcolor[rgb]{0.10,0.09,0.49}{#1}}
\newcommand{\VerbatimStringTok}[1]{\textcolor[rgb]{0.25,0.44,0.63}{#1}}
\newcommand{\WarningTok}[1]{\textcolor[rgb]{0.38,0.63,0.69}{\textbf{\textit{#1}}}}

\providecommand{\tightlist}{%
  \setlength{\itemsep}{0pt}\setlength{\parskip}{0pt}}

% =================================================================
% --- 3. COLORS & GRAPHICS ---
% =================================================================
\usepackage{graphicx}
\usepackage{float}

\definecolor{brandOrange}{RGB}{\BrandColorRGB}
\definecolor{mediumOrange}{RGB}{250, 206, 156} 
\definecolor{lightOrange}{RGB}{253, 226, 202} 

\makeatletter
\def\maxwidth{\ifdim\Gin@nat@width>\linewidth\linewidth\else\Gin@nat@width\fi}
\def\maxheight{\ifdim\Gin@nat@height>\textheight\textheight\else\Gin@nat@height\fi}
\makeatother
\setkeys{Gin}{width=\maxwidth, height=\maxheight, keepaspectratio}

% Fix hình ảnh căn giữa (Không cần dịch chuyển nữa vì lề trái phải đã bằng nhau 2.5cm)
\let\origfigure\figure
\let\endorigfigure\endfigure
\renewenvironment{figure}[1][H] {\origfigure[#1]\centering} {\endorigfigure}

% =================================================================
% --- 4. HEADER & FOOTER STYLING ---
% =================================================================
\pagestyle{fancy}
\fancyhf{} 

% Mở rộng Header sang trái
\fancyhfoffset[L]{\NumberWidth}

% Header
% Header
    \fancyhead[L]{\fontfamily{ptm}\selectfont\textcolor{gray}{\small }}       % XÓA \vspace{3pt}

% Rules
\renewcommand{\headrulewidth}{0.4pt}
\setlength{\headheight}{15pt} % <--- Có thể giảm thêm nếu cần thiết (ví dụ: 12pt)
\renewcommand{\headrule}{\hbox to\headwidth{\color{gray}\leaders\hrule height \headrulewidth\hfill}}

% Footer
\fancyfoot[L]{\fontfamily{ptm}\selectfont\textcolor{gray}{\small Copyright \textcopyright\ \CopyrightYear, \CompanyCopyright.}}
\fancyfoot[R]{\fontfamily{ptm}\selectfont\thepage}

% Rules
\renewcommand{\headrulewidth}{0.4pt}
\setlength{\headheight}{15pt} 
\renewcommand{\headrule}{\hbox to\headwidth{\color{gray}\leaders\hrule height \headrulewidth\hfill}}

\renewcommand{\footrulewidth}{0pt} 

% =================================================================
% --- 5. HEADING STYLES ---
% =================================================================
\usepackage{titlesec}

\newcommand{\hangnum}[1]{\llap{\makebox[\NumberWidth][l]{#1}}}

\titlespacing*{\section}{0pt}{12pt}{6pt}
\titlespacing*{\subsection}{0pt}{10pt}{4pt}
\titlespacing*{\subsubsection}{0pt}{8pt}{3pt}

\titleformat{\section}[hang]
  {\color{brandOrange}\normalfont\Large\bfseries} 
  {\hangnum{\thesection.}} 
  {0pt} {}

\titleformat{\subsection}[hang]
  {\color{black}\normalfont\large\bfseries}
  {\hangnum{\thesubsection.}} 
  {0pt} {}

\titleformat{\subsubsection}[hang]
  {\color{black}\normalfont\normalsize\bfseries}
  {\hangnum{\thesubsubsection.}} 
  {0pt} {}

\providecommand{\e}{}
\providecommand{\s}{\textbackslash s}
\providecommand{\pandocbounded}[1]{#1}
\usepackage{etoolbox}


% =================================================================
% --- 6. LIST STYLES ---
% =================================================================
\usepackage{enumitem}
\usepackage{amssymb} 

\setlist{nosep, leftmargin=*} 
\setlist[itemize,1]{label=\raisebox{-0.9ex}{\textcolor{black}{\Huge\textbullet}}}
\setlist[itemize,2]{label=\textcolor{black}{\tiny$\blacksquare$}}
\setlist[itemize,3]{label=\textbullet}

% =================================================================
% --- 7. TABLES & CAPTIONS ---
% =================================================================
\usepackage{array}
\usepackage{longtable}
\usepackage{caption}

\captionsetup{
  labelfont={bf, color=black},
  textfont={it},
  labelsep=colon,
  skip=10pt,
  position=bottom 
}

\newcommand{\globalTableCaption}{}
\AtBeginEnvironment{longtable}{
  \renewcommand{\globalTableCaption}{} 
  \let\OldCaption\caption 
  \renewcommand{\caption}[2][]{
     \renewcommand{\globalTableCaption}{##2}
     \addcontentsline{lot}{table}{\numberline{\thetable}##2}
  }
}
\AtEndEnvironment{longtable}{
  \ifdefempty{\globalTableCaption}{}{
    \begin{center} \captionof{table}{\globalTableCaption} \end{center}
  }
}

% =================================================================
% --- 8. TOC & HYPERLINKS ---
% =================================================================
\usepackage{tocloft}
\usepackage[unicode=true]{hyperref}

\tocloftpagestyle{fancy} 

\renewcommand{\cfttoctitlefont}{%
    \hspace{-\NumberWidth}%
    \color{brandOrange}\Huge\bfseries\sffamily\MakeUppercase
}
\renewcommand{\cftlottitlefont}{%
    \hspace{-\NumberWidth}%
    \color{brandOrange}\Huge\bfseries\sffamily\MakeUppercase
}
\renewcommand{\cftloftitlefont}{%
    \hspace{-\NumberWidth}%
    \color{brandOrange}\Huge\bfseries\sffamily\MakeUppercase
}

\setlength{\cftsecindent}{-\NumberWidth}    
\setlength{\cftsubsecindent}{0pt}           
\setlength{\cftsubsubsecindent}{1.5em}      

\hypersetup{
    colorlinks=true, linkcolor=blue, filecolor=magenta, urlcolor=blue,
    pdftitle={Controlled Document Convention}, pdfpagemode=FullScreen,
}

\newcommand{\insertTOC}{\newpage \renewcommand{\contentsname}{TABLE OF CONTENT} \tableofcontents \newpage}
\newcommand{\insertListTables}{\newpage \renewcommand{\listtablename}{INDEX OF TABLES} \listoftables \newpage}
\newcommand{\insertListFigures}{\newpage \renewcommand{\listfigurename}{INDEX OF FIGURES} \listoffigures \newpage}


% =================================================================
% --- 9. TITLE PAGE (TRANG BÌA) --- [FIXED CENTER]
% =================================================================
\usepackage{epstopdf} % Gói này hỗ trợ xử lý hình ảnh
\graphicspath{{./}{../}{Bestarion.QM.CON.001-Controlled_Document_Convention-2_1/}}
\renewcommand{\maketitle}{
    \thispagestyle{empty}
    
    \begin{titlepage}
        % Cấu hình lề cho trang bìa
        \newgeometry{
            top=3cm, bottom=3cm, 
            left=2.5cm, right=2.5cm
        }
        
        % Đảm bảo không có thụt đầu dòng nào ảnh hưởng
        \setlength{\parindent}{0pt}
        \centering 
        
        \vspace*{1cm}
        
        % Logo
        \IfFileExists{logo.jpeg}{\includegraphics[width=4cm]{logo.jpeg}}{
            \IfFileExists{logo.png}{\includegraphics[width=4cm]{logo.png}}{
                \begin{center}\fbox{\parbox[c][2cm]{4cm}{\centering LOGO MISSING}}\end{center}
            }
        }
        \vspace{0.5cm} 
        
        % Title
        {\color{brandOrange} \fontsize{30}{75}\selectfont  \par}
        
        % Subtitle
        
        
        \vfill 
        
        % Metadata Table
        % Dùng minipage để gói gọn bảng metadata và căn giữa nó
        \begin{minipage}{0.8\textwidth} 
            \centering % Căn giữa nội dung trong minipage (nếu cần) hoặc để raggedright
            \raggedright
            \renewcommand{\arraystretch}{1.5}
            \begin{tabular}{ll}
                \textbf{Security classification:} & \detokenize{INTERNAL} \\
                \textbf{Document code:} & \detokenize{} \\
                \textbf{Last updated by:} & \detokenize{Khoa
Nguyen-Dinh} \\
                \textbf{Effective date:} & \detokenize{Dec 10, 2025} \\
                \textbf{Version:} & \detokenize{1.0} \\
                %\textbf{Template ID:} & \detokenize{ODT\_Base\_Template} \\
                \textbf{Template ID:} & ODT\_Base\_Template \\
            \end{tabular}
        \end{minipage}
        
        \restoregeometry
    \end{titlepage}
}

% =================================================================
% --- 10. DOCUMENT CONTENT --- [ĐÃ SỬA]
% =================================================================
\onehalfspacing
\begin{document}

\maketitle 
\clearpage 

% --- Document Control Table ---
\noindent\hspace{-\NumberWidth}%
{\color{brandOrange}\fontsize{15}{18}\selectfont\bfseries DOCUMENT CONTROL}%
\par\vspace{6pt}

% Table này sẽ làm bộ đếm tăng lên 1
\begin{longtable}{|p{0.09\textwidth}|p{0.22\textwidth}|p{0.14\textwidth}|p{0.12\textwidth}|p{0.16\textwidth}|p{0.12\textwidth}|}
    \hline
    \rowcolor{\TableHeadColor}
    \textbf{Version} & \textbf{Change Description} & \textbf{Changed By} & \textbf{Date} & \textbf{Approved By} & \textbf{Date} \\
    \hline
        1.0 & Khởi tạo tài
liệu & KhoaND & 2025-12-10 &  & 2025-12-10 \\ \hline
    \end{longtable}
\addtocounter{table}{+1}
\clearpage 

% --- TOC ---
\insertTOC  
\insertListTables  
\insertListFigures

% >>> QUAN TRỌNG: RESET BỘ ĐẾM TRƯỚC KHI VÀO NỘI DUNG CHÍNH <<<
% Đặt lại đếm bảng về 0 (để bảng tiếp theo sẽ là Table 1)
\setcounter{table}{0}
\setcounter{figure}{0}

% --- Main Body ---
\section{\texorpdfstring{\textbf{Overview}}{Overview}}\label{overview}

\subsection{\texorpdfstring{\textbf{Purpose}}{Purpose}}\label{purpose}

This document provides comprehensive guidance for deploying an EC2
instance in the AWS nonprod testing environment using Terraform
Infrastructure as Code (IaC). The instance will host Apache Kafka,
Redis, and related services with proper networking, security controls,
and access management. The deployment follows AWS best practices and
maintains consistency with the existing MAX project infrastructure.

\subsection{\texorpdfstring{\textbf{Scope}}{Scope}}\label{scope}

This guide encompasses the complete deployment lifecycle for other
services infrastructure using Terraform. It provides step-by-step
instructions for provisioning compute resources, configuring security
controls, setting up networking integration, and managing the deployment
through the S3 remote state backend.

The scope does NOT include: - VPC and networking infrastructure creation
- Load balancing or auto-scaling setup

\subsection{\texorpdfstring{\textbf{Abbreviations and
Definitions}}{Abbreviations and Definitions}}\label{abbreviations-and-definitions}

\begin{itemize}
\item
  \textbf{EC2}: Elastic Compute Cloud - AWS cloud computing service that
  provides virtual servers. This is where Kafka, Redis, and related
  services run.
\item
  \textbf{VPC}: Virtual Private Cloud - A private virtual network in AWS
  that allows users to create a separate network space with complete
  control over IP addressing and routing.
\item
  \textbf{AMI}: Amazon Machine Image - A pre-configured virtual machine
  image used to launch EC2 instances. Contains the operating system,
  applications, and services needed.
\item
  \textbf{CIDR}: Classless Inter-Domain Routing - An IP routing method
  that allows for efficient allocation and management of IP address
  ranges using prefix notation.
\item
  \textbf{IAM}: Identity and Access Management - AWS service that
  manages user access and permissions for AWS resources.
\item
  \textbf{SSH}: Secure Shell - A cryptographic network protocol for
  secure data communication and remote command execution.
\item
  \textbf{SSM}: Systems Manager - AWS service that provides a unified
  interface for managing and automating operational tasks across AWS
  resources.
\item
  \textbf{EBS}: Elastic Block Store - A scalable block storage service
  for use with EC2 instances, providing persistent storage.
\end{itemize}

\section{\texorpdfstring{\textbf{Architecture}}{Architecture}}\label{architecture}

\subsection{\texorpdfstring{\textbf{Diagram}}{Diagram}}\label{diagram}

\begin{figure}
\centering
\pandocbounded{\includegraphics[keepaspectratio,alt={Architecture EC2}]{./images/ec2.png}}
\caption{Architecture EC2}
\end{figure}

\begin{figure}
\centering
\pandocbounded{\includegraphics[keepaspectratio,alt={Architecture Network}]{./images/network.png}}
\caption{Architecture Network}
\end{figure}

\subsection{\texorpdfstring{\textbf{Directory
Structure}}{Directory Structure}}\label{directory-structure}

\begin{verbatim}
/nonprod/testing/other-services/
├── .terraform.lock.hcl
├── main.tf
├── versions.tf
└── outputs.tf
\end{verbatim}

\section{\texorpdfstring{\textbf{Requirements}}{Requirements}}\label{requirements}

The Terraform configuration must meet the following requirements:

\begin{enumerate}
\def\labelenumi{\arabic{enumi}.}
\item
  \textbf{Network Integration}: The EC2 instance must be placed in the
  same network as other services for proper communication and resource
  sharing. (managed in \texttt{live/nonprod/general/networking})
\item
  \textbf{Operating System Image}: Use the latest Ubuntu AMI available
  at the time of deployment for current security patches and features.
\item
  \textbf{CIDR Ranges}: Configure access from two CIDR ranges:

  \begin{itemize}
  \tightlist
  \item
    CIDR block containing current resources
  \item
    CIDR block of the MSS project
  \end{itemize}
\item
  \textbf{Resource Tagging}: Apply consistent tags to all resources for
  management and governance:

  \begin{itemize}
  \tightlist
  \item
    \textbf{Environment}: Deployment environment designation
  \item
    \textbf{Project}: Project identifier
  \item
    \textbf{Owner}: Responsible party or team
  \item
    \textbf{Service}: Service name
  \end{itemize}
\item
  \textbf{Security Group Configuration}: Security groups must be
  properly configured to restrict access to only the required ports from
  authorized CIDR ranges, enhancing overall security posture.
\item
  \textbf{Outbound Internet Access}: The EC2 instance must have outbound
  internet connectivity to download OS updates, security patches, and
  required software packages.
\item
  \textbf{Instance Naming Convention}: EC2 instances must follow the
  established naming convention for easy identification and management.
\item
  \textbf{Instance Type Selection}: Choose an appropriate instance type
  that balances resource requirements for Kafka, Redis, and related
  services while optimizing for cost efficiency.
\item
  \textbf{IAM Role Configuration}: Configure the IAM role to ensure the
  EC2 instance has necessary permissions to access AWS Systems Manager
  and other required AWS services.
\item
  \textbf{Private Subnet Deployment}: Deploy the EC2 instance in a
  private subnet to enhance security and prevent direct internet
  exposure.
\item
  \textbf{SSH Key Pair}: Reuse the existing key pair for SSH access,
  ensuring consistency across the nonprod environment.
\end{enumerate}

\section{\texorpdfstring{\textbf{Prerequisites}}{Prerequisites}}\label{prerequisites}

Before running the Terraform configuration, verify the following
prerequisites are met:

\subsection{\texorpdfstring{\textbf{Terraform and AWS Provider
Requirements}}{Terraform and AWS Provider Requirements}}\label{terraform-and-aws-provider-requirements}

\begin{center}
{\renewcommand{\arraystretch}{1.5}
\begin{tabular}{|l|l|}
\hline

\rowcolor{brandOrange}\textbf{Name} & \textbf{Version} \\
\hline

terraform & >= 1.5.0 \\

\hline

aws & >= 5.0 \\

\hline

\end{tabular}}

\captionof{table}{abc}

\end{center}

\subsubsection{\texorpdfstring{\textbf{Providers}}{Providers}}\label{providers}

\begin{center}
{\renewcommand{\arraystretch}{1.5}
\begin{tabular}{|l|l|}
\hline

\rowcolor{brandOrange}\textbf{Name} & \textbf{Version} \\
\hline

aws & 6.22.1 \\

\hline

terraform & n/a \\

\hline

\end{tabular}}

\end{center}

\subsection{\texorpdfstring{\textbf{Prerequisites}}{Prerequisites}}\label{prerequisites-1}

\begin{enumerate}
\def\labelenumi{\arabic{enumi}.}
\tightlist
\item
  \textbf{VPC and Subnets}: Verify that the VPC and required subnets
  exist and are properly configured.\\
\item
  \textbf{VPC Configuration}: Confirm that VPC ID and subnet name tags
  match the Terraform configuration.\\
\item
  \textbf{IAM Permissions}: Ensure your AWS credentials have all
  required EC2, S3, and IAM permissions for resource creation.\\
\item
  \textbf{AMI Availability}: Verify that the specified AMI ID is valid
  and available in the region.\\
\item
  \textbf{AWS Systems Manager Role}: Verify that the
  AmazonSSMRoleForInstancesQuickSetup IAM role exists in your AWS
  account.\\
\item
  \textbf{SSH Key Pair}: Verify that the key pair exists in the region
  for EC2 instance access.\\
\item
  \textbf{Terraform Module}: Verify that the EC2 module is available at
  the correct path.\\
\item
  \textbf{CIDR Range Validation}: Confirm that both CIDR ranges are
  properly configured and accessible.\\
\item
  \textbf{AWS CLI Connectivity}: Ensure your machine has network
  connectivity to AWS APIs and can authenticate with your AWS
  credentials.\\
\item
  \textbf{Remote State Backend}: Verify that the S3 backend and
  networking state file exist for remote state management.
\end{enumerate}

\section{\texorpdfstring{\textbf{Deployment
Instructions}}{Deployment Instructions}}\label{deployment-instructions}

\subsection{\texorpdfstring{\textbf{Deploy
Infrastructure}}{Deploy Infrastructure}}\label{deploy-infrastructure}

\subsubsection{\texorpdfstring{\textbf{Step 1: Initialize Terraform
Workspace}}{Step 1: Initialize Terraform Workspace}}\label{step-1-initialize-terraform-workspace}

Navigate to the Terraform directory and initialize the workspace:

\begin{Shaded}
\begin{Highlighting}[]
\BuiltInTok{cd}\NormalTok{ /nonprod/testing/other{-}services}
\ExtensionTok{terraform}\NormalTok{ init}
\end{Highlighting}
\end{Shaded}

This command downloads required providers and initializes the backend.

\subsubsection{\texorpdfstring{\textbf{Step 2: Validate Configuration
Syntax}}{Step 2: Validate Configuration Syntax}}\label{step-2-validate-configuration-syntax}

Verify the Terraform configuration syntax:

\begin{Shaded}
\begin{Highlighting}[]
\ExtensionTok{terraform}\NormalTok{ validate}
\end{Highlighting}
\end{Shaded}

Expected output: \texttt{Success!\ The\ configuration\ is\ valid.}

\subsubsection{\texorpdfstring{\textbf{Step 3: Format Code
(Optional)}}{Step 3: Format Code (Optional)}}\label{step-3-format-code-optional}

Format Terraform files for consistency:

\begin{Shaded}
\begin{Highlighting}[]
\ExtensionTok{terraform}\NormalTok{ fmt }\AttributeTok{{-}recursive}
\end{Highlighting}
\end{Shaded}

\subsubsection{\texorpdfstring{\textbf{Step 4: Generate Execution
Plan}}{Step 4: Generate Execution Plan}}\label{step-4-generate-execution-plan}

Create a detailed plan of infrastructure changes:

\begin{Shaded}
\begin{Highlighting}[]
\ExtensionTok{terraform}\NormalTok{ plan }\AttributeTok{{-}out}\OperatorTok{=}\NormalTok{tfplan}
\end{Highlighting}
\end{Shaded}

Review the plan:

\begin{Shaded}
\begin{Highlighting}[]
\ExtensionTok{terraform}\NormalTok{ show tfplan}
\end{Highlighting}
\end{Shaded}

\subsubsection{\texorpdfstring{\textbf{Step 5: Apply
Configuration}}{Step 5: Apply Configuration}}\label{step-5-apply-configuration}

Deploy infrastructure according to the plan:

\begin{Shaded}
\begin{Highlighting}[]
\ExtensionTok{terraform}\NormalTok{ apply tfplan}
\end{Highlighting}
\end{Shaded}

Expected output:
\texttt{Apply\ complete!\ Resources:\ 9\ added,\ 0\ changed,\ 0\ destroyed.}

\subsubsection{\texorpdfstring{\textbf{Step 6: Retrieve
Outputs}}{Step 6: Retrieve Outputs}}\label{step-6-retrieve-outputs}

Display infrastructure outputs:

\begin{Shaded}
\begin{Highlighting}[]
\ExtensionTok{terraform}\NormalTok{ output}
\end{Highlighting}
\end{Shaded}

\subsubsection{\texorpdfstring{\textbf{Step 7: Retrieve SSH Private
Key}}{Step 7: Retrieve SSH Private Key}}\label{step-7-retrieve-ssh-private-key}

Get the private key for SSH access:

\begin{Shaded}
\begin{Highlighting}[]
\VariableTok{RETRIEVE\_CMD}\OperatorTok{=}\VariableTok{$(}\ExtensionTok{terraform}\NormalTok{ output }\AttributeTok{{-}raw}\NormalTok{ retrieve\_private\_key\_command}\VariableTok{)}
\BuiltInTok{eval} \VariableTok{$RETRIEVE\_CMD}
\end{Highlighting}
\end{Shaded}

\subsubsection{\texorpdfstring{\textbf{Step 8: Connect to EC2
Instance}}{Step 8: Connect to EC2 Instance}}\label{step-8-connect-to-ec2-instance}

Establish SSH connection to the instance:

\begin{Shaded}
\begin{Highlighting}[]
\VariableTok{INSTANCE\_IP}\OperatorTok{=}\VariableTok{$(}\ExtensionTok{terraform}\NormalTok{ output }\AttributeTok{{-}raw}\NormalTok{ instance\_private\_ip}\VariableTok{)}
\FunctionTok{ssh} \AttributeTok{{-}i}\NormalTok{ max.dev.key.01.pem ec2{-}user@}\VariableTok{$INSTANCE\_IP}
\end{Highlighting}
\end{Shaded}

\subsection{\texorpdfstring{\textbf{Redeploying After Accidental
Destruction}}{Redeploying After Accidental Destruction}}\label{redeploying-after-accidental-destruction}

If the EC2 instance is accidentally terminated outside of Terraform:

\begin{Shaded}
\begin{Highlighting}[]
\CommentTok{\# Refresh Terraform state}
\ExtensionTok{terraform}\NormalTok{ refresh}

\CommentTok{\# Check state}
\ExtensionTok{terraform}\NormalTok{ state list}

\CommentTok{\# Reapply to recreate the instance}
\ExtensionTok{terraform}\NormalTok{ apply }\AttributeTok{{-}auto{-}approve}
\end{Highlighting}
\end{Shaded}

\subsection{\texorpdfstring{\textbf{Destroying and Recreating
Infrastructure}}{Destroying and Recreating Infrastructure}}\label{destroying-and-recreating-infrastructure}

To completely remove and redeploy:

\begin{Shaded}
\begin{Highlighting}[]
\CommentTok{\# Review resources to be destroyed}
\ExtensionTok{terraform}\NormalTok{ plan }\AttributeTok{{-}destroy} \AttributeTok{{-}out}\OperatorTok{=}\NormalTok{tfplan\_destroy}
\ExtensionTok{terraform}\NormalTok{ show tfplan\_destroy}

\CommentTok{\# Destroy all resources}
\ExtensionTok{terraform}\NormalTok{ apply tfplan\_destroy}

\CommentTok{\# Recreate from scratch}
\ExtensionTok{terraform}\NormalTok{ apply }\AttributeTok{{-}auto{-}approve}
\end{Highlighting}
\end{Shaded}

\textbf{Critical Warning}: This will permanently delete all resources
including the EC2 instance and security groups.

\subsection{\texorpdfstring{\textbf{Applying Partial Changes (Targeted
Apply)}}{Applying Partial Changes (Targeted Apply)}}\label{applying-partial-changes-targeted-apply}

To apply changes to specific resources only:

\begin{Shaded}
\begin{Highlighting}[]
\CommentTok{\# Example: Update only the security group without touching EC2}
\ExtensionTok{terraform}\NormalTok{ apply }\AttributeTok{{-}target}\OperatorTok{=}\NormalTok{aws\_security\_group.other\_services}

\CommentTok{\# Example: Update specific security group rule}
\ExtensionTok{terraform}\NormalTok{ apply }\AttributeTok{{-}target}\OperatorTok{=}\StringTok{\textquotesingle{}aws\_security\_group\_rule.allow\_kafka\textquotesingle{}}

\CommentTok{\# Update EC2 module only}
\ExtensionTok{terraform}\NormalTok{ apply }\AttributeTok{{-}target}\OperatorTok{=}\StringTok{\textquotesingle{}module.ec2\textquotesingle{}}
\end{Highlighting}
\end{Shaded}

\textbf{Note}: Use targeted applies with caution as it may cause
inconsistencies.

\subsection{\texorpdfstring{\textbf{Backup Current State Before Major
Changes}}{Backup Current State Before Major Changes}}\label{backup-current-state-before-major-changes}

Before making significant changes, backup the current state:

\begin{Shaded}
\begin{Highlighting}[]
\CommentTok{\# Backup current state locally}
\FunctionTok{cp}\NormalTok{ terraform.tfstate terraform.tfstate.backup.}\VariableTok{$(}\FunctionTok{date}\NormalTok{ +\%Y\%m\%d\_\%H\%M\%S}\VariableTok{)}

\CommentTok{\# Or view current state}
\ExtensionTok{terraform}\NormalTok{ state show}

\CommentTok{\# Verify remote state in S3}
\ExtensionTok{aws}\NormalTok{ s3 cp s3://meperia{-}edi/terraform/testing/other{-}services/terraform.tfstate terraform.tfstate.remote}
\end{Highlighting}
\end{Shaded}

\subsection{\texorpdfstring{\textbf{Reviewing and Approving Changes
(Plan Review
Process)}}{Reviewing and Approving Changes (Plan Review Process)}}\label{reviewing-and-approving-changes-plan-review-process}

For change control and approval:

\begin{Shaded}
\begin{Highlighting}[]
\CommentTok{\# Generate detailed plan in JSON format for review}
\ExtensionTok{terraform}\NormalTok{ plan }\AttributeTok{{-}out}\OperatorTok{=}\NormalTok{tfplan }\AttributeTok{{-}json} \OperatorTok{\textgreater{}}\NormalTok{ tfplan.json}

\CommentTok{\# Generate human{-}readable summary}
\ExtensionTok{terraform}\NormalTok{ plan }\AttributeTok{{-}out}\OperatorTok{=}\NormalTok{tfplan }\OperatorTok{\textgreater{}}\NormalTok{ tfplan.txt}

\CommentTok{\# Share plan files for review and approval before applying}
\CommentTok{\# Once approved, proceed with apply}
\ExtensionTok{terraform}\NormalTok{ apply tfplan}
\end{Highlighting}
\end{Shaded}

\section{\texorpdfstring{\textbf{Appendix}}{Appendix}}\label{appendix}

\subsubsection{\texorpdfstring{\textbf{Modules}}{Modules}}\label{modules}

\begin{center}
{\renewcommand{\arraystretch}{1.5}
\begin{tabular}{|l|l|l|}
\hline

\rowcolor{brandOrange}\textbf{Name} & \textbf{Source} & \textbf{Version} \\
\hline

ec2 & ../../../../modules/ec2 & n/a \\

\hline

\end{tabular}}

\end{center}

\subsubsection{\texorpdfstring{\textbf{Resources}}{Resources}}\label{resources}

\begin{center}
{\renewcommand{\arraystretch}{1.5}
\begin{tabular}{|l|l|}
\hline

\rowcolor{brandOrange}\textbf{Name} & \textbf{Type} \\
\hline

aws\_security\_group.other\_services & resource \\

\hline

aws\_security\_group\_rule.allow\_kafka & resource \\

\hline

aws\_security\_group\_rule.allow\_kafka\_ui & resource \\

\hline

aws\_security\_group\_rule.allow\_redis & resource \\

\hline

aws\_security\_group\_rule.allow\_ssh & resource \\

\hline

aws\_subnet.private\_app\_a & data source \\

\hline

terraform\_remote\_state.networking & data source \\

\hline

\end{tabular}}

\end{center}

\subsubsection{\texorpdfstring{\textbf{Inputs}}{Inputs}}\label{inputs}

No inputs required.

\subsubsection{\texorpdfstring{\textbf{Outputs}}{Outputs}}\label{outputs}

\begin{center}
{\renewcommand{\arraystretch}{1.5}
\begin{tabular}{|l|l|}
\hline

\rowcolor{brandOrange}\textbf{Name} & \textbf{Description} \\
\hline

ebs\_volume\_id & ID of the attached EBS volume \\

\hline

effective\_key\_name & The key name associated with the EC2 instance \\

\hline

instance\_id & ID of the created EC2 instance \\

\hline

instance\_private\_ip & Private IP of the EC2 instance \\

\hline

private\_key\_parameter\_name & SSM Parameter name storing the generated private key (if generated) \\

\hline

retrieve\_private\_key\_command & AWS CLI command to retrieve the private key from SSM Parameter Store \\

\hline

\end{tabular}}

\end{center}

\subsection{\texorpdfstring{\textbf{Troubleshooting}}{Troubleshooting}}\label{troubleshooting}

\subsubsection{\texorpdfstring{\textbf{Common
Issues}}{Common Issues}}\label{common-issues}

\begin{itemize}
\tightlist
\item
  \textbf{Backend State Not Found}: Verify S3 bucket and networking
  state file exist
\item
  \textbf{VPC/Subnet Not Found}: Confirm VPC ID and subnet name tags
  match configuration
\item
  \textbf{IAM Permissions Denied}: Verify IAM role has required
  permissions
\item
  \textbf{AMI Not Available}: Verify AMI ID is valid and available in
  region
\item
  \textbf{Key Pair Not Found}: Ensure SSH key pair exists in AWS region
\end{itemize}

\subsection{\texorpdfstring{\textbf{References}}{References}}\label{references}

For additional information, refer to:

\begin{itemize}
\tightlist
\item
  Terraform AWS Provider Documentation
\item
  AWS EC2 User Guide
\item
  AWS VPC Best Practices
\item
  Security Group Configuration Guide
\end{itemize}

\end{document}