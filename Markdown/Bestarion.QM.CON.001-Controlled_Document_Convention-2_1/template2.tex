\documentclass[11pt,a4paper]{article}

% =================================================================
% --- 1. CÁC GÓI CƠ BẢN & CẤU HÌNH TRANG ---
% =================================================================
\usepackage[utf8]{inputenc}
\usepackage{geometry}
\usepackage[english]{babel}
\usepackage{fontspec}
\usepackage{setspace}

% Các gói hỗ trợ code block và màu sắc
\usepackage{color}
\usepackage{fancyvrb}
\usepackage{framed}

% Các gói toán học và bảng
\usepackage{amsmath}
\usepackage{amssymb}
\usepackage{calc}
\usepackage{array}
\usepackage{longtable}
\usepackage{float}

% Gói hỗ trợ logic
\usepackage{etoolbox}



% >>> THÊM ĐOẠN NÀY ĐỂ FIX LỖI SHADED (PANDOC HIGHLIGHTING) <<<
% >>> KẾT THÚC ĐOẠN THÊM <<<

% Cấu hình Font chữ (Ưu tiên Arial, nếu không có thì dùng TeX Gyre Heros)
\IfFontExistsTF{Arial}{
    \setmainfont{Arial}
}{
    \setmainfont{TeX Gyre Heros}
}

% Xử lý khoảng cách đoạn văn
\providecommand{\tightlist}{%
  \setlength{\itemsep}{0pt}\setlength{\parskip}{0pt}}

% =================================================================
% --- 2. MÀU SẮC & HEADER / FOOTER ---
% =================================================================
\usepackage[table]{xcolor}
\usepackage{fancyhdr}

% Định nghĩa màu Cam thương hiệu
\definecolor{brandOrange}{RGB}{237, 125, 49}

% Cấu hình Header và Footer
\pagestyle{fancy}
\fancyhf{} 

% Header Trái: Lấy từ biến header_line hoặc Title
    \fancyhead[L]{\textcolor{gray}{\small Bestarion.QM.CON.001-Controlled\_Document\_Convention-2\_1}}

% Footer Phải: Số trang
\fancyfoot[R]{\thepage}

% Đường kẻ Header màu xám
\renewcommand{\headrulewidth}{0.4pt}
\renewcommand{\headrule}{\hbox to\headwidth{\color{gray}\leaders\hrule height \headrulewidth\hfill}}

% =================================================================
% --- 2.1. CẤU HÌNH TIÊU ĐỀ (HEADING STYLE) --- [MỚI]
% =================================================================
\usepackage{titlesec}

\titlespacing*{\section}{0pt}{12pt}{6pt}
\titlespacing*{\subsection}{0pt}{10pt}{4pt}
\titlespacing*{\subsubsection}{0pt}{8pt}{3pt}

% YÊU CẦU 1: Heading 1 (Section) màu cam
\titleformat{\section}
  {\color{brandOrange}\normalfont\Large\bfseries} % Format: Màu cam, Font thường, Lớn, In đậm
  {\thesection.}{1em}{}

% Giữ nguyên các cấp khác (đen mặc định)
\titleformat{\subsection}
  {\color{black}\normalfont\large\bfseries}
  {\thesubsection.}{1em}{}
  
\titleformat{\subsubsection}
  {\color{black}\normalfont\normalsize\bfseries}
  {\thesubsubsection.}{1em}{}

% =================================================================
% --- 2.2. CẤU HÌNH DANH SÁCH (BULLET POINTS) --- [MỚI]
% =================================================================
\usepackage{enumitem}

% YÊU CẦU 2: Tùy chỉnh Bullet Point các cấp
% Cấp 1: Dấu tròn to hơn (\Large \textbullet)
\setlist[itemize,1]{label=\raisebox{-0.9ex}{\textcolor{black}{\Huge\textbullet}}}

% Cấp 2: Hình vuông full đen (\blacksquare) - Cần gói amssymb đã nạp ở trên
\setlist[itemize,2]{label=\textcolor{black}{\tiny$\blacksquare$}}

% Cấp 3: Dấu tròn thường (\textbullet)
\setlist[itemize,3]{label=\textbullet}

% Tùy chỉnh khoảng cách cho đẹp hơn (tùy chọn)
\setlist{nosep} % Giảm khoảng cách giữa các dòng

% =================================================================
% --- 3. CẤU HÌNH BẢNG (TABLE) ---
% =================================================================

% Định nghĩa màu Cam thương hiệu (Đã có, H1)
\definecolor{brandOrange}{RGB}{237, 125, 49} 
% Định nghĩa màu Cam trung bình (H2)
\definecolor{mediumOrange}{RGB}{250, 206, 156} 
% Định nghĩa màu Cam nhạt (H3)
\definecolor{lightOrange}{RGB}{253, 226, 202} 

\providecommand{\toprule}{}
\providecommand{\midrule}{}
\providecommand{\bottomrule}{}
\renewcommand{\toprule}{\relax}
\renewcommand{\midrule}{\relax}
\renewcommand{\bottomrule}{\relax}

% =================================================================
% --- 4. CẤU HÌNH HÌNH ẢNH (FIGURE) ---
% =================================================================
\usepackage{graphicx}
\usepackage{float}

% Logic TỰ ĐỘNG THU NHỎ ẢNH
\makeatletter
\def\maxwidth{\ifdim\Gin@nat@width>\linewidth\linewidth\else\Gin@nat@width\fi}
\def\maxheight{\ifdim\Gin@nat@height>\textheight\textheight\else\Gin@nat@height\fi}
\makeatother
\setkeys{Gin}{width=\maxwidth, height=\maxheight, keepaspectratio}

% Ép hình ảnh luôn căn giữa
\let\origfigure\figure
\let\endorigfigure\endfigure
\renewenvironment{figure}[1][H] {
    \origfigure[#1]
    \centering
} {
    \endorigfigure
}

% =================================================================
% --- 5. CẤU HÌNH CAPTION (CHÚ THÍCH) ---
% =================================================================
\usepackage{caption}

\captionsetup{
  labelfont={bf, color=black},
  textfont={it},
  labelsep=colon,
  skip=10pt,
  position=bottom 
}

% Ép caption bảng xuống dưới
\newcommand{\globalTableCaption}{}

\AtBeginEnvironment{longtable}{
  \renewcommand{\globalTableCaption}{} 
  \let\OldCaption\caption 
  \renewcommand{\caption}[2][]{
     \renewcommand{\globalTableCaption}{##2}
     \addcontentsline{lot}{table}{\numberline{\thetable}##2}
  }
}

\AtEndEnvironment{longtable}{
  \ifdefempty{\globalTableCaption}{}{
    \begin{center} 
       \captionof{table}{\globalTableCaption}
    \end{center}
  }
}

% =================================================================
% --- 5.1. BỔ SUNG FIX LỖI ---
% =================================================================
% Các gói fix lỗi đã được nạp ở phần 1 (calc, booktabs...)
\newcounter{none}

% Định nghĩa các lệnh đặc biệt
\providecommand{\e}{}
\providecommand{\s}{\textbackslash s}
\providecommand{\pandocbounded}[1]{#1}

% =================================================================
% --- 6. CẤU HÌNH MỤC LỤC (TOC, LOT, LOF) ---
% =================================================================
\usepackage{tocloft}

\tocloftpagestyle{fancy} 

% --- A. MỤC LỤC CHÍNH ---
\renewcommand{\cfttoctitlefont}{\color{brandOrange}\Huge\bfseries\sffamily\MakeUppercase}
\renewcommand{\cftsecfont}{\sffamily}
\renewcommand{\cftsubsecfont}{\sffamily}
\renewcommand{\cftsecleader}{\cftdotfill{\cftdotsep}}

% --- B. MỤC LỤC BẢNG ---
\renewcommand{\cftlottitlefont}{\color{brandOrange}\Huge\bfseries\sffamily\MakeUppercase}
\renewcommand{\cfttabfont}{\sffamily}
\renewcommand{\cfttabpagefont}{\sffamily}
\renewcommand{\cfttableader}{\cftdotfill{\cftdotsep}}

% --- C. MỤC LỤC HÌNH ẢNH ---
\renewcommand{\cftloftitlefont}{\color{brandOrange}\Huge\bfseries\sffamily\MakeUppercase}
\renewcommand{\cftfigfont}{\sffamily}
\renewcommand{\cftfigpagefont}{\sffamily}
\renewcommand{\cftfigleader}{\cftdotfill{\cftdotsep}}

% =================================================================
% --- 7. CÁC LỆNH CHÈN THỦ CÔNG ---
% =================================================================

\newcommand{\customTitle}[1]{
    \par\noindent 
    {\color{brandOrange}\Huge\bfseries\sffamily\MakeUppercase{#1}}
    \vspace{0.5cm} 
    \par
}

\newcommand{\insertTOC}{
    \newpage
    \renewcommand{\contentsname}{TABLE OF CONTENT} 
    \tableofcontents
    \newpage
}

\newcommand{\insertListTables}{
    \newpage
    \renewcommand{\listtablename}{INDEX OF TABLES} 
    \listoftables
    \newpage
}

\newcommand{\insertListFigures}{
    \newpage
    \renewcommand{\listfigurename}{INDEX OF FIGURES} 
    \listoffigures
    \newpage
}

% =================================================================
% --- 8. HYPERLINKS & METADATA ---
% =================================================================
\usepackage[unicode=true]{hyperref}
\hypersetup{
    colorlinks=true,
    linkcolor=blue,
    filecolor=magenta,      
    urlcolor=blue,
    pdftitle={Controlled Document Convention},
    pdfpagemode=FullScreen,
}

% =================================================================
% --- 9. TRANG BÌA (TITLE PAGE) ---
% =================================================================
\renewcommand{\maketitle}{
    \begin{titlepage}
        \newgeometry{top=3cm, bottom=3cm, left=2.5cm, right=2.5cm}
        \centering
        \vspace*{1cm}
        \IfFileExists{logo.jpeg}{\includegraphics[width=4cm]{logo.jpeg}}{
            \IfFileExists{logo.png}{\includegraphics[width=4cm]{logo.png}}{
                \begin{center}\fbox{\parbox[c][2cm]{4cm}{\centering LOGO MISSING}}\end{center}
            }
        }
        \vspace{0.5cm} 
        
        {\color{brandOrange} \fontsize{30}{75}\selectfont CONTROLLED
DOCUMENT CONVENTION \par}
         \vspace{1.3cm} {\fontsize{24}{28}\selectfont QM \par} 
        
        \vfill 
        \raggedright
        \renewcommand{\arraystretch}{1.5}
        
        \begin{tabular}{ll}
            \textbf{Security classification:} & \detokenize{INTERNAL} \\      % <-- THÊM \detokenize
            \textbf{Document code:} & \detokenize{Bestarion.QM.CON.001} \\              % <-- THÊM \detokenize
            \textbf{Last updated by:} & \detokenize{Khoa
Nguyen-Dinh} \\                        % Tên người thường ít khi có _, nhưng nếu có hãy bọc luôn
            \textbf{Effective date:} & \detokenize{Aug 09, 2023} \\
            \textbf{Version:} & \detokenize{2.1} \\
            \textbf{Template ID:} & \detokenize{ODT\_Base\_Template} \\
        \end{tabular}
        \restoregeometry
    \end{titlepage}
}

\newenvironment{ReviewTable}[1] % [1] là tham số cho caption
{
    % Bắt đầu longtable với 6 cột. p{0.3\textwidth} cho cột mô tả.
    % {|c|p{0.3\textwidth}|c|c|c|c|}
    \begin{longtable}{|c|p{0.3\textwidth}|c|c|c|c|}
    \caption{#1} \\ 
    \hline
    \rowcolor{brandOrange}
    
    % Hàng Tiêu đề Cố định (6 cột)
    \textbf{Version} & 
    \textbf{Change description} & 
    \textbf{Changed by} & 
    \textbf{Date} & 
    % Column 5: Dùng \makecell để xuống dòng
    \makecell{\textbf{Reviewed} \\ \textbf{/Approved by}} & 
    \textbf{Date} \\ 
    \hline
    
    \endfirsthead

    % Tiêu đề lặp lại (Đảm bảo 6 cột)
    \multicolumn{6}{c}{\textbf{Bảng \thetable{} -- Tiếp theo}} \\ % Sửa \multicolumn thành 6 cột
    \hline
    \rowcolor{brandOrange}
    \textbf{Version} & 
    \textbf{Change description} & 
    \textbf{Changed by} & 
    \textbf{Date} & 
    \makecell{\textbf{Reviewed} \\ \textbf{/Approved by}} & 
    \textbf{Date} \\ 
    \hline
}
{
    \end{longtable}
}

% =================================================================
% --- 10. NỘI DUNG CHÍNH --- [ĐÃ SỬA: PHÂN TÁCH TIÊU ĐỀ]
% =================================================================
\onehalfspacing
\begin{document}
\maketitle 

% 1. TIÊU ĐỀ MỤC (Lấy từ CONTROLLED DOCUMENT
CONVENTION - Tên tài liệu đầy đủ)
\customTitle{CONTROLLED DOCUMENT CONVENTION}

% 2. BẢNG KIỂM SOÁT (Lấy từ )
    \begin{ReviewTable}{DOCUMENT CONTROL TABLE} % Fallback nếu không có biến
    % ... (Dữ liệu bảng sẽ được chèn từ Markdown)
\end{ReviewTable}
\clearpage 

% Các mục lục (TOC, LOT, LOF)
\insertTOC  
\insertListTables  
\insertListFigures

% NỘI DUNG CHÍNH TỪ MARKDOWN
\hypertarget{introduction}{%
\section{\texorpdfstring{\textbf{Introduction}}{Introduction}}\label{introduction}}

\hypertarget{purpose}{%
\subsection{\texorpdfstring{\textbf{Purpose}}{Purpose}}\label{purpose}}

This document specifies conventions for naming and version of documents
which are in scope of QMS \& ISMS.

\hypertarget{scope}{%
\subsection{\texorpdfstring{\textbf{Scope}}{Scope}}\label{scope}}

This convention is applied to only the controlled documents, which are
internally created by Bestarion.

Following documents are outside the scope of this convention:

\begin{itemize}
\tightlist
\item
  Uncontrolled documents and records
\item
  Documents obtained from external sources (such as, government
  regulatory,\ldots)
\item
  Project documents to follow customers' conventions, regarding naming
  and/ or version
\end{itemize}

\hypertarget{definition-and-abbreviation}{%
\subsection{\texorpdfstring{\textbf{Definition And
Abbreviation}}{Definition And Abbreviation}}\label{definition-and-abbreviation}}

\begin{itemize}
\item
  BoD: Board of Directors
\item
  QM: Quality Management
\item
  Controlled Document: A document with its versions to be controlled, in
  order to ensure appropriate
\end{itemize}

access to correct content. When a controlled document is to be changed,
its new

version must be controlled via appropriate review and approval
procedures

Document: Information and its supporting medium (text files, images,
video, voice,\ldots)

Record: Is a kind of document which is resulted from performing
operations

\hypertarget{references}{%
\subsection{\texorpdfstring{\textbf{References}}{References}}\label{references}}

\begin{itemize}
\tightlist
\item
  {[}1{]} Document\_Responsibility\_Authority\_Standard
\item
  {[}2{]} Control\_Of\_Documented\_Information
\end{itemize}

\hypertarget{regulation}{%
\section{\texorpdfstring{\textbf{Regulation}}{Regulation}}\label{regulation}}

The objective for document owner when creating a new document is no more
than 1 easily identifiable error in 5 A4 pages

\hypertarget{naming-of-controlled-non-record-documents}{%
\section{\texorpdfstring{\textbf{Naming Of Controlled Non-record
Documents}}{Naming Of Controlled Non-record Documents}}\label{naming-of-controlled-non-record-documents}}

When a controlled document is not a record, the file name shall follow
below format:

\textbf{{[}Document Code{]}-{[}Document Name{]}-{[}Version Number{]}}

Each part is separated by a hyphen.

\hypertarget{document-code}{%
\subsection{\texorpdfstring{\textbf{Document
Code}}{Document Code}}\label{document-code}}

The document code can be used as a short name to refer to the document.

Document Code is constructed by following format:

\textbf{{[}Unit Code{]}.{[}Document Type Code{]}.{[}Ordinal Number{]}}

Each part is separated by a stop (`.').

\begin{itemize}
\tightlist
\item
  \textbf{Unit Code}: Code representing the unit

  \begin{itemize}
  \tightlist
  \item
    In case unit is a department or portfolio: refer to corporate
    organizational structure for the corresponding code

    \begin{itemize}
    \tightlist
    \item
      Bestarion

      \begin{itemize}
      \tightlist
      \item
        IT: Information Technology Department
      \end{itemize}
    \end{itemize}
  \item
    In case unit is a sub-group within a department or a portfolio:
    refer to the department/ portfolio charter for the corresponding
    code
  \item
    In case document apply in a sub-company, unit code may be included
    company name
  \end{itemize}
\item
  \textbf{Document Type Code}: Refer to
  \protect\hyperlink{page-5-8}{{[}1{]}} for the codes of document types
\item
  \textbf{Ordinal Number} (3 digits): A unique number to differentiate
  documents in the same document type, belonging to the same unit
\end{itemize}

\emph{Examples: HR.GUI.234, QM.POL.001},
\emph{Bestarion.IT.POL.001}\ldots{}

\hypertarget{document-name}{%
\subsection{\texorpdfstring{\textbf{Document
Name}}{Document Name}}\label{document-name}}

Document name is a unique name of document, with following convention:

\begin{itemize}
\tightlist
\item
  The first letter of each word shall be in capital
\item
  Words are separated by underscore character (`\_')
\end{itemize}

\emph{Examples: Project\_Planning, Risk\_Management.}

Document name should not include the type of document, since the type is
already known in Document Code part \protect\hyperlink{page-5-1}{(3.1.)}

\hypertarget{version-number}{%
\subsection{\texorpdfstring{\textbf{Version
Number}}{Version Number}}\label{version-number}}

A version number is constructed of 3 numbers: ``Major version number'',
``Minor version number'' and ``Draft number''.

Those numbers are separated by a underscore (``\_'').

\textbf{{[}Major Version Number{]}\_{[}Minor Version Number{]}\_{[}Draft
Number{]}}

\begin{itemize}
\tightlist
\item
  \textbf{Major version number}: Started from 0. The first baseline has
  major version equals 1. Major version is increased when there are
  changes to the document content, which affect to the original meaning
  of the document
\item
  \textbf{Minor version number}: Started from 0. Minor version is
  increased when there are only changes to enhance or correct the
  content, while the original meaning is intact
\item
  \textbf{Draft number}: Started from 1. This number is increased when a
  draft is updated and re-submitted for review. A document may pass
  through several draft updates until it is baseline
\item
  \textbf{Note:} Before the first baseline of each document, draft
  version number for document is constructed of 2 numbers: ``Major
  version number'' and ``Draft number''
\item
  \emph{Examples:}

  \begin{itemize}
  \tightlist
  \item
    \emph{QM.POL.001-Quality\_Manual-0\_1 The first draft of policy
    `Quality Manual', issued by QM}
  \item
    \emph{QM.POL.001-Quality\_Manual-0\_2 The second draft, re-submitted
    for review}
  \end{itemize}
\end{itemize}

\begin{longtable}[]{@{}lll@{}}
\caption{Testing image embedding}\tabularnewline
\toprule
\begin{minipage}[b]{0.03\columnwidth}\raggedright
\strut
\end{minipage} & \begin{minipage}[b]{0.30\columnwidth}\raggedright
QM.POL.001-Quality\_Manual-1\_0\strut
\end{minipage} & \begin{minipage}[b]{0.58\columnwidth}\raggedright
The first baseline of policy `Quality Manual', issuedby QM\strut
\end{minipage}\tabularnewline
\midrule
\endfirsthead
\toprule
\begin{minipage}[b]{0.03\columnwidth}\raggedright
\strut
\end{minipage} & \begin{minipage}[b]{0.30\columnwidth}\raggedright
QM.POL.001-Quality\_Manual-1\_0\strut
\end{minipage} & \begin{minipage}[b]{0.58\columnwidth}\raggedright
The first baseline of policy `Quality Manual', issuedby QM\strut
\end{minipage}\tabularnewline
\midrule
\endhead
\begin{minipage}[t]{0.03\columnwidth}\raggedright
\strut
\end{minipage} & \begin{minipage}[t]{0.30\columnwidth}\raggedright
QM.POL.001-Quality\_Manual-1\_0\_1\strut
\end{minipage} & \begin{minipage}[t]{0.58\columnwidth}\raggedright
The first draft version after the first baseline\strut
\end{minipage}\tabularnewline
\begin{minipage}[t]{0.03\columnwidth}\raggedright
\strut
\end{minipage} & \begin{minipage}[t]{0.30\columnwidth}\raggedright
QM.POL.001-Quality\_Manual-1\_1\strut
\end{minipage} & \begin{minipage}[t]{0.58\columnwidth}\raggedright
The minor baseline of the document, based onversion 1.0\strut
\end{minipage}\tabularnewline
\begin{minipage}[t]{0.03\columnwidth}\raggedright
\strut
\end{minipage} & \begin{minipage}[t]{0.30\columnwidth}\raggedright
QM.POL.001-Quality\_Manual-2\_0\strut
\end{minipage} & \begin{minipage}[t]{0.58\columnwidth}\raggedright
The major baseline of the document, based onversion 1.x\strut
\end{minipage}\tabularnewline
\bottomrule
\end{longtable}

\end{document}